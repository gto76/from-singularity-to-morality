\documentclass[10pt]{book}
\usepackage{graphicx}
\usepackage{anysize}
\usepackage{geometry}
\geometry{papersize={12.7cm, 19.5cm}}
\marginsize{1cm}{1.5cm}{0.4cm}{0.6cm}

\begin{document}

\title{FROM SINGULARITY TO MORALITY}
\author{Jure \v Sorn}
\maketitle

\tableofcontents

%=============================
%=============================
%CHAPTER 1
\chapter {From Object to Subject}

Ok...

\section{So Hegel says about basic ordinary Knowing that...}

So I lost it. Not enough of a frame for me, but nonetheless, let's not get too pessimistic. If we take into account pursuit of Hegel's philosophy to resolve everything to its end, how come it's still able to posses structure instead of crumbling into itself?
I think one must attribute this to the gaps; so this leads us to the figure-ground / chicken-or-the-egg problem; what is more important, what was first: the philosophy/structure/wissenschaft or the gaps/the incomprehensible?
As interesting and absolute as this problem seems it may be of little importance in itself unless we can render it to something a bit more 'earthly'; for instance: How does it relate to morals/"primordial good/evil"/...
But to return to the notion of crumbling. When does the philosophy/wissenschaft begin to crumble into itself, and when is it standing; does it not all the time 'contain' this crumbling (or at least the notion of it) in itself? Is not the concept of Now nothing but this crumbling into itself?
So the question is how does it manage to maintain, despite the crumbling which seems to be the essence of it? Well, it must be due to this unconnectable gap between now and the last/next now; because although they are the same nows, and they stand one by the other, they just cannot be connected, merged or a little shifted, because of the notion of greater Now they impose. This notion of Now takes one of them and makes it a subject while other remains the object, and so although they remain the same, the multitude of them makes them different, and thus multiple; and we are back at the chicken or the egg problem. So we get this structure, but without the zero point, without the anchor. It's like if 2D fractals actually have somewhere between 1 and 2 dimensions and if "Thought" has only one dimension (that of cause and effect), then this structure lives somewhere between 0 and 1 dimension (effect effecting the cause). 
So with this established let's look at how it renders itself in physics. There we get instead of one, two ends of the thread. One micro and other macroscopic: neutrinos and galaxies. So far, if we look at the material world as less pure than the spiritual (ideological), it makes some sense to have more ends. It also makes sense that the closer we get to the ends, the purer and at the same time more contradictory the forms get. But the question that poses itself here is: Is there the gap between the ends and the forms or do the forms gradually morph into the void? And since this is a 'lesser' world, will it tell us anything about the (non)existence of the primordial gap in the philosophical domain?

\footnote{
Interesting correspondence of religious thoughts and scientific views of the universe:
1. Infinitely expanding universe - Juda
2. Static universe - Paga
3. Collapsing universe - Buda
}

Of course the last question smells, but let's examine the different possible answers that Physics as such is capable of giving us: Discrete vs continuous basis, Singular vs multiple universes, Infinite vs finite universe... So it gets clear pretty quickly that physical explanations of world at its ends automatically contain the spiritual (ideological) aspects.
But nonetheless (or because of it), the big question for/on both sides is the existence of the zero point and its relation to the void/non-existence. Is zero point same as void? But how can a point be a void at the same time? Is void not the opposite of a point, no matter from how far we look? If there is no zero point, then why does it then seem like everything is pointing toward it?
To get a bit more metaphysical; Is not the first and the only thing such entity as zero point can "do", to destroy itself by splitting into two parts, which of course cannot anymore be zero points, but one "positive" and other "negative" part of it? But since this jump looks like a necessity is it not still in a way the zero point. I would say no, because it already incorporates something like a "decision": Which part becomes positive and which negative. This might seem like just a problem of naming things, but from where do we then get this notion of positive/negative. If it is from some later "event" down the line, how was then this primordial jump even possible?
But nonetheless maybe we are moving too fast here, specifically: Even though the primordial entity splits itself into two parts, and by definition this parts cannot be equal, because that would mean the entity multiplied itself, which would be impossible at this stage (because of lack of space for two of them:) and thus it can only split itself into two opposing entities, the "sum" of which is same as that of the entity itself; Nonetheless does the notion that they are opposing already imply that one of them specifically is positive? Maybe, but it certainly dozen't imply which, because they are also the same. Of course here we are already not only in domain of quarks, but also in part that of electrons. Although an electron cannot freely change into a proton; nonetheless the electrons themselves are indistinguishable among themselves - one is totally the same as the other (so it is meaningless to discuss for instance how any electrons that were once part of Buddha on average does a person contain in its body) - what keeps them in "order" is law of conservation. But again we are moving to fast here. 
We should examine the nature of the two Opposites some more. Precisely their being of the same matter, same ancestor, same cause, conceived at the same moment and still being different, maybe more accurate; being opposite. The question is from where does this concept of opposition stem. Is it already inherent in the Entity itself, or does it somehow emerge by itself from the impossibility of the Entity to "do" anything? Is thus the Opposition not in opposition to the Entity(in itself/from itself)? The moment it emerges is itself in opposition to the Entity, but since this would mean the surplus, the original Entity must at the same time vanish, thus leaving the Opposition by itself -> what renders the two opposites, but also the opposition in the situation itself (namely parts being the same, but opposite at the same time), which is the real surplus we get from this split/jump. 
A quick and jumpy conclusion from this would be that, thus the wheels are set in motion from the start and the Jump is in it's nature an discrete event and thus the singularity unattainable by gradual deduction. And that the most primordial thing/idea that is accessible to us is that of opposition. And that it would actually enable us to grasp the singularity in its entirety if it was not for the opposition that it poses to itself by making the same at same time opposite. This surplus sets the wheel in motion, which makes it impossible to observe the singularity, also because we need the surplus at the first place to observe. -> This of course nicely parallels itself to nirvana experience, in which we have to lose more and more "stuff" until at the end we also have to let go of the observer.

\section{Good and Evil as primordial opposites..}

So the question that poses itself here is: What is the connection between good and evil and this two primordial opposites (and their surplus)?
In a way good and evil are the highest abstractions for the humans. We use them to substitute for us what is most important to us, although we always have trouble trying to label "stuff" with one or the other. We are also frequently questioning is it at all possible to bucket stuff this way; nonetheless the notion of Good and Evil is as present as ever. 
So to tackle it systematically, we have basically two possibilities: Are the two Opposites themselves good, because they themselves together alone are the Entity, and the surplus that stems from them evil? (Buddha) Or the other way around? (Judo/Christian - glorification of falling into)
Let's say that we identified the anchor of two different (opposite?) mortalities. How do we then get from this anchor to the other less general rules of morality (10 commandment, Buddhist basic morality,...), which are if we look at the world actually more general and present throughout cultures. Namely: Don't: kill, steal, be greedy, (angry, delusional), lie, slander.. One explanation would be that these rules are helping in maintaining larger societies and thus help in propagating themselves. So they are but the stage for the deepest morality which is one of the primordial ones, namely Buddhas: Don't conduct in sexual misconduct (excessive passion) vs Biblical Thou shalt not commit adultery and Thou shalt have no other gods which of course also serve the cause of propagating and same is with Buddha.
So by accepting this explanation we basically diminish those laws to the simple universal state laws, but at the same time we see between their lines their true meaning, namely taking sides in one of the primordial opposites. Two things of interest arise from here. First is if the majority of morality is strictly utilitarian in sens of trying to propagate itself, how would morality look in a world where opposite actions are necessary for a morality to propagate itself; namely: killing, stealing, lying.. (of course inside of community itself).  

But more interestingly, last distinction between essential/primordial part of morality and its utilitarian/reproductive gives us a chance to parallel (them) the two moralities to the two oppositions and their excess. This is getting a little confusing. Again: We have two opposing primordial moralities, and they form the excess which manifests itself in common moralities, which the two share. So if we look from Judeo-Christian perspective to this structure, we favor falling in/sticking with them/embracing them, and from Buddhist perspective we try to "acquire dispassion for them". So from both perspectives, (common) moralities stand for a kind of a ladder, the only question is which direction is up, where should we climb? So are we to put them on the altar, embrace them fully again and again, let them shape our existence and follow where it leads us (thus basically abandoning their original context), or should we grow more and more dispassion for them, not abandoning them, but paving our path with them, thus giving less and less thought to them as we move forward with their help?   

Conclusion from above could be: Although the utilitarian moralities might seem like lesser ones comparing to the primordial, they are still moralities and thus in a way indivisible from each other. Although the judo and Buddha moralities are basically leading in opposite ways, it would be hard for one to label another as evil, since evil is supposed to be what lies outside moralities and both share commons which are indivisible from primordial part according to moralities. 
But if we assume both of them to be focused on their path, we can't but notice the very different perspective that is unfolding in front of them. To the Buddhist one, as it is moving along the road paved from basic moralities, the picture of primordial split is becoming ever clearer. And not until one can fully accept this split can one reach nirvana and thus basically abandon the road that let him achieve it. There is of course little need for god in this perspective.
The Judeo-Christian perspective, at the other hand, is facing endless road ahead, disappearing in haze. The possibilities for her are endless. Of course from this open position then stem endless controversies such as determinism/free will and god..  

\section{A little appendix to the section}

[err: remove Buda:, juda:]
BUDA: More we move this way, less lies outside morality -> we get enclosed by morality itself, but still, to reach singularity we have to cut off the excess, thus staying only with Buddha and Juda, which by this cut become the same. So the perspective before the cut would be very clear view of double opposite moralities.
JUDA:\footnote{From now on juda stands for judeo-christian.} The other way that has no end (it is Bodhisattvan by it's nature), it is surrounded by less and less morality, basically soaked in particularities, thus in a way penetrating deeper and deeper into god (void) (party getting out of hand)
Morality is in a way the last thing (frontier) that is of us, what is truly godly is outside morality, the void, the incomprehensible, the Random. In other way, closing in to nirvana, god, Random, the incomprehensible is of less and less importance. 
Thus the evil perceived as "stuff" outside moralities holds in the middle, but further up or down we go, we need a different approach to classification, (just) because of practicality.. In case of Buda side, Juda as evil would make sense, of course up until nirvana (still not answer to what can go wrong in nirvana domain) In case of Juda, almost fully emerged in god..., probably not not embracing more could be sign of evil (but that could also hold for Buda) But better question would be what is good? In a way notion disappears and we HAVE to trust god.
And is the third way the paganism - standing still on the leather?

How to explain, by acquired means, the Tibetan position to the question of needlessness (unnecessity/unnecessariness) of morality to reach nirvana, that something can go terribly wrong in nirvana domain itself?
The way to start tackling the problem probably lies in the substitution of words by \v Zizek, from evil to 'terribly wrong'. It signifies the shift from our "earthly" moralities to some higher, but still maintaining the negative signifier. So instead of asking what could go wrong with nirvana domain, let's ask instead what does it mean for something to go wrong there, or rather what can even go wrong there, or even what can "go on" there? Is nirvana not suppose to imply stasis?
First paradox here arises, that Zizek is trying to tackle a problem proposed by Tibetan tradition, within a scope of Theravada tradition. Particularly he is suggesting a problem in nirvana domain, but transplanted from a tradition with a different view of nirvana itself. Namely don't Tibetan Buddhist count the Bodhisattvas to be operating in nirvana domain itself. This domain seems different from Theravadean domain of nirvana which although after obtaining it you stay here and nothing changes is in a way purer shift than obtaining it and than returning. Theravadean nirvana thus does not leave you an option of getting back, precisely because you did not went anywhere, where at the other hand one could imagine an Bodhisattva having an possibility of turning back and becoming "evil".
%---------
%---------
For \v Zizek absolute knowledge (p48) is Lacanian pass\'e: 
The final moment of the analytic process, "The experience of the lack of other" (p27). So in path of Juda this would be an Christian enlightenment - The death of god) So the moment god almost takes up completely it dies (the moment we or I attain absolute knowledge) And so through god everything is permitted (as long as one of course makes sure this path can be propagated (reproduced for others) -> as long as the framework, "empty" customs maintain -> catholic church 
So here seems that Juda differs from Buda: namely in the necessity of path -> where Buda is free to abandon the "path" once it attains the enlightenment (at least in Theravada), to Juda the conservation of the "path" after achieving enlightenment seems more important. [ UNDEVELOPED] But still as absolute knowledge is obtained through judo enlightenment, it can in a way let go of primordial singularity -> it in a way becomes a new singularity.. where as nirvana is obtained it can let go of all the dualities. - [Here it seams that as I am operating with this absolute concepts still too attached to the notion of self, as if one point is moving around the field, what can of course be useful analogy when operating around the middle, but as we are approaching extremes, the self is getting under attack.]
Absolute knowledge and nirvana as opposites(dualities).
In a way they don't contradict, they don't produce excess and thus they succeed in what the primordial duality "failed" at, namely creating stable duality (without excess) But another reading could also be that only together they compose absolute knowledge?? But this would render Juda and Buda enlightenment the same, would it not?  One can not attain the absolute absolute knowledge, one that would include absolute knowledge and nirvana together, precisely because they together form the stable duality, one without excess. And it is this lack of excess that "prohibits" the continuation of dialectical process, that would lead to next step, namely that of absolute absolute knowledge.

%=============================
%=============================
%CHAPTER 2 
\chapter {Nature of Being and the World}

1.How deep must we fall, before we can turn around and start climbing toward nirvana?
2.Do we attain the Object when we reach nirvana? Or rather do "we" become on Object when we reach nirvana?
3.And how come that the dialectic stops somewhere (at absolute knowledge)? And is there any property in primordial singularity (on it?) that is giving us the clue that the dialectic will have an end? (and does the number of levels have any significance? 5?)
%????????????????

A dot on time is line, a line in time is surface, a surface in time is volume, a volume in time is entity > we have 5 levels -> all we need to do now is get rid of time's temporal property, namely getting rid of past, present, future and we attain absolute knowledge.
Of course the dot is not a primordial singularity, even if in space of 0 dimensions, because it possesses its opposite > 0 dimension without a dot, namely the dimension itself. They are in a way primordial dualities, who's opposition produces excess, they resolve into the 1D space, since only way they can coexist is on a line (in time).
So we could ask here why not 5 dimensions? Does Hegel's dialectic have 5 levels because we have 5 fingers? Or because we live in 3 dimensions.
See how simple things as sunrise/set influence the basis of our belief! Namely Buda in east, Juda in west, and Hindu in middle (because India has both east and west shore, and paganism being to local to have any shore, and also Tibet (and Mongolia) having no shore) (is Hinduism the all encircling way?) When you are on a shore looking toward east or west, one convenient way is to continue your journey the same way, and other that as civilization, you see yourself as a center/most advanced, that was always pushing in this way until it reached the coast, and that before us is only one more big jump. 

\section{Which property of singularity is telling us, that the dialectic will end?}

It is probably the sign of the first sentence: "let's not get to pessimistic. If we take into account pursuit of Hegel's philosophy to resolve everything to its end, how come it's still able to posses structure instead of crumbling into itself." So this is then the hook that we were looking for, not just absolute knowledge, that is the realization, but the remark, that we can come from the faceless singularity to the realization itself. It seems illogical, but if we first discard time, which we have to attain absolute knowledge, then the realization may appear to us as a necessity. On the more philosophical level, although that we always expected to discover some "magic" at the end, when we do, it still seams impossible, although that it is in it's purest, nonrepresentational form the only way of magic to be, namely from faceless singularity to ending dialectic. Only other "magic-less" way would be for singularity to not explode. So in a way that excess of a jump renders itself in a finiteness of dialectic. (But we could probably also reverse it: So in a way that jump renders itself in a excess of finiteness of dialectic). To put it other way, the "magic" of a jump cancels itself with a "magic" of end of dialectic. Since we started with impossibility, is not the only logical way that we end with one. 
So here we get a new duality, that between the jump and the end of dialectic. But they are not opposite!!! How do they correspond to the duality of singularity and dialectic (absolute knowledge). They are in a way endpoints, one of singularity and other of (primordial) duality / dialectic, but are also the endpoint of a stasis and of process/anti-stasis (only that the endpoint of process does not lead back to primordial singularity, but to s stasis of a process/form) But the fact is they are not circular. the end of a process does not lead back to stasis (true/primordial singularity), nor thus it lead to new process; It just becomes the true (and only possible) opposite of it (primordial singularity). So loss is not perceived as such anymore. 

How deep must we fall, so we can start climbing back toward nirvana?
Simple answer: Deep enough that Buddhism can form. So probably enlightened religion? So all the way to the last step before absolute knowledge (?) So the entry point to enlightenment for both Juda and Buda is at the same end edge of progress, namely the enlightened religion, only that one offers a step out ant another a slide back into the beginning?

To take it afresh, why is the end of dialectical process essential?
Of course the singularity and duality/absolute knowledge do not posses in themselves the opposite (really?). For absolute knowledge, the singularity is lost and for singularity the absolute knowledge is unattainable. They are separated by the jump. So from this distant perspective jump loses it's directionality and becomes just a barrier. (we just happen to find ourselves on this side) So is the question how did we come here (find ourselves on this side)? No! The question that makes sense to us should be: Should we embrace it or escape it (distance ourselves from it) (the side). So we can look at the two paths (given to individual) as the only true alternatives, that in a way can't produce an synthesis, and thus as only true ethics. But of course one can choose to follow one half of his life one and the other the second. But not I think, if one follows Juda, he in a way never stops; he embraces the embracement and so on.. The moment one stops, and turns around, he becomes unethical.
And when one reaches nirvana, well he stays in this world, and can teach others and offer them support,but what are his ethics after nirvana?

Both ways of course require us of letting go of the pictorial representation and of notion of present/past/future, thus both ways must let go of a ethics (morals) that require final judgment, or even the notion of it (even if unattainable). One ethics that can do without it. [err ???] Well, does he who becomes just the observer of a samsara, not loose ethics as such. Don't ethics only exist in a domain of absolute knowledge? [err ???] But when one obtains Juda enlightenment, does he step out? No, he is out of enlightened religion, but still inside absolute knowledge. [err ???]
So is this level attainable for someone who is trying to reach nirvana?
In a way ethics is all that remains in judo enlightenment (ethics of ethics) (and thus they consume morals) and totally disappear in Buda enlightenment. (In both cases morals are probably the first victim) [So only ethics of the commandments remain at this point, but this is probably one step before the enlightenment] > probably the ethics of Juda enlightenment > nirvana as abandonment of ethics > is this even bearable for human/spirit > 
Does one abandon spirit too (along with ethics)?

So if we arrived to the end in a spiritual domain, will this render itself as achieving the end in hard sciences too, namely physics? Will we finally succeed in getting rid of time? And will it not lead us to the same end as philosophy, namely the finding of the magic, but that magic being the only possible one, so in a way a bit of disappointment for us. Will it not lead us to the duh.. moment. (Like what did you expect? Something to end, and not be magical at the same time?) But let's dwell a little more here. How specifically would that be stated in physics?
So the magic of a center/big bang requires a magic of its edge/end/finiteness. The universe can not be infinite, or at least have infinite mass/energy. As effect without a cause the impossibility of a jump/big bang requires the impossibility of a finite mass/energy and the impossibility of a finite mass/energy requires an impossibility of a jump/big bang. 
And since big bang and big collapse are from a distant perspective, without a temporal dimension, basically the different manifestations of the same thing, namely the primordial jump/gap, we are only left with two concepts: gap and here (universe) and thus singularity is forever lost for us, but the loss enabled our existence, so we get at least that :)
[err>] So the wrong way of predicting what that would manifest itself in physics is that the system (of equations) is the singularity we were looking for, although we will never find the missing part of equation, because it does not exist, so in a way we will be forced to put in a most imminent spot of a final equation one blank variable (lambda, one that can be substituted with any equation, thus also with itself.), and thus we will conclude the circle; Finish and connect the system.
(That would probably stand from Gödel/Escher/Bach's perspective, but not from Hegel's. But even from his/theirs perspective, are not fractals bounded on one side; they have surface (at least some), are not infinitely big; they only have infinite edge.. Gödel/Escher/Bach as Hegel for the Anglo-Saxons.. Do they think the path is infinite?) But this would probably mean the endless recursion and thus endless dialectic. [>err] More Hegelian would be stance, that we will find the last equation, but that alone will imply, that something must have been left out. We still won't be able to explain big bang/collapse, but this exactly will require big bang/collapse to have happened, this will be it's explanation.
So to put it in a way less dramatically: One can't get a big bang/collapse withouts a closed system of equations, and one can not get closed system of equations without a big bang/collapse.\footnote{Question that still stands is how come did this equations and system become so balanced?}
Both are irregularities, that "cancel" each other out, or better constitute the underlying composure that both sides need, although they are entirely separated.
So even singularity in a most primordial sense needs (or already possesses in its self) our universe, although it may seem self-sufficient.\footnote{So another question that remains is: Why do we then perceive singularity as self-sufficient?} So to put it simply, we are to accept the impossibility of true/stable primordial singularity. But are we? What about nirvana? Is not nirvana as such, obtaining this or at least some singularity?

Science:
So will not equation at the end connect (make sense), but their exclusion of a big bang/collapse will not make sense, so we will get that it was lost from beginning [big if]

Magic:
At the end when we find it, is that act of finding it not exactly the magic itself, and thus the only imaginable way of finding one. More precisely, the magic that we find is not the object (singularity) itself, but the fact that we are able to accept, that it was lost from the beginning. The possibility of our search ending without finding it is what is magical (or the actual magic) (and incomprehensible.)
So in a way that two magics are then in a way true duality, one in which singularity tried to split itself into, but failed. True (stable) dualities in sense that they are opposite, but not opposing, the true two faces of a coin, that do not generate excess. So in some way they are the two bounds of our world. (In physics finite time and finite space, or better that of micro and macro) So on one side magic of the gap bounds us from the singularity/object and at the other the magic of attainability of absolute knowledge bounds us from the infinite other, from the void. (So in a more physical language big bang preventing matter to get infinitely small (disappearing) and a system of equations/laws, preventing matter to be infinite) If this bound would be missing and the equations would not be closable, it would probably mean our universe would be a fractal...
So with an act of splitting, singularity succeeds in bounding itself (buying herself structure, for the price of substance) Although she was self sufficient, she was missing structure, and although we have structure, we are missing substance. So in a way she exchanged her being/essence/center for structure, she turned herself inside out, got rid of a substance. (Like Johnson sells its soul to the devil for the success; changes his substance for structure)

Religion not seeing itself as the other "magical" boundary. 
Two magical boundaries (that are of the same) [are they? this seems too jumpy of a conclusion], which give rise to structure without substance - the onion.
So to put it very simplistically, we are at the problem of figure ground, and singularity changes from a dot to enclosing space, where structure is possible, but even more simplistically and maybe accurately: a dot that is not in space and thus in-itself voids itself, by becoming a circle (shape without surface), and thus with loosing its surface enables a structure to emerge in-itself [is this not pictorial thinking?, cut this paragraph drastically]. But even more mathematically correct, a dot (not a sphere) expands itself into a circle in one step (what is morphologically impossible; It should first become a line, which then connects its ends - that way former analogy is better)

[cocaine vs LSD]
magic:
So the solution that would not require a magic would be one of recursion, namely that electrons would be just a little planets, and so on (universe of infinite detail and size), and since this is not true, we absolutely need little "magic". So this "magic" is precisely the magic of possibility of primordial jump and that of possibility of absolute knowledge, which are two sides of a coin. So through the magic of absolute knowledge/end of dialectic, we realize the magic of primordial jump/gap, that it really happened. So singularity exchanged it's infinity that it possessed, for structure. So it seized to be total, for the ability to be creative. [Here we look at it trough form of a story; of course more singular/absolute way of looking at it would be that, singularity already possesses in it's essence structure] [but it also probably bounds itself with time; but this seams a little high price to pay for a jump - it would be OK if it would endlessly expand and contract, so to give some dynamic structure to it]

Which religions/believes do not have the end of time? That would explain an non-urgency to go either way on the leather.
So the magic of the jump implies, that there exists the smallest "object" "under" which there is nothing
So the circularity of a universe (time) seems the only logical system in which we don't "bound" time, but still maintain its temporal/action giving component.
But this of course rises the question of what would happen the moment one universe is not contactable. This would of course stop the cycle. So a better solution would be a infinity of universes, all going of at the same time.
And out of this course comes big question: Can we cheat it? Can we influence in any way the next cycle? [a bit too sci-fi, but also maybe not relevant?] Are we ourselves (our cycle) a product of this kind of cheating? But more importantly, what would this "cheating"/gluing together/transcending of the cycles mean for the time and universe itself?
And if there can not be any influence, does this mean that all the cycles are exactly the same (that would be kind of disappointing. Of course here rises the question of determinism: Are all cycles the same or do they take up all possible forms? The later way gives rise to interesting parallel to quantum physics: So in a way we have free will, but looking at the big picture, our choice ends up just as a part of a graph of normal distribution. So in a way it is known beforehand how regularly your act will occur, only it was not known that it will happen precisely in this cycle. Or better: we will choose one way and thus have free will as long as there exists an observer, otherwise our actions will spread nicely trough specter - we will choose both ways simultaneously (if we reduce choice to two paths). The question that of course remains is are we enough of a observer to count:)

[put next three paragraphs in proper place]
Space:
So the space (void, other) itself can be infinite, but it doesn't mater to us as long as mass/energy/structure [absolute knowledge] is. 

One could also draw a parallel between a spiritual and material world as one representing primordial singularity and other the other side. But that would then mean: 
a twofold explosion (weird). According to this ideas exist before big bang, but the need for it seizes!!! Or maybe just that for us the big other is our universe. -> Better not to mix "stuff"/dimensions for now.

Drive: So you don't see "drive" in a universe. Maybe it was here only in the beginning, taking form of a inflation.

%===============================================================
%===============================================================
%CHAPTER 3
\chapter{Juda vs Buda}

Does Jump/end duality correlate to singularity/absolute knowledge duality? What is their relation? Probably they are figure ground. [Develop it further!]jump-end = big bang/collapse-micro/macro barrier.
1. Get rid of time: bang = collapse
2. So we get three: bang/collapse, micro, macro
3. Other: space + time
4. Absolute knowledge: final mass/energy + finite cycle

SIMPLISTIC:
So since we live in a physical world, which is lesser than the spiritual, we have two others, and thus four barriers, two for every other.
PROBLEMATIC:
So bang/collapse are of the same, but micro/macro are not. (We could probably still have finite mass with infinite detail, and infinite mass without infinite detail.) But with time: time without end and time without beginning equal in length to time without both. They are all infinite.
So how does this two correspond? (singularity and absolute knowledge vs jump and end)

How did we get time? [It seems mandatory for thing to happen, but still] We don't need it in spiritual world. It seams rational that we need it, but from where did this "extra" dimension came? [From where did any dimension come? From dialectic?] 
%----
Zizek talks about symmetry within nirvana... So nirvana is not obtaining primordial singularity of course, but primordial duality!!! Duh. So a state without a consciousness. (nirvana = object a)
%----
-Buda's and Juda's image of each other, and a possibility of a middle ground (Buda is already middle way).
-Almodovar: All about my mother.
%?????????????????????????????
					Buda		Juda
Intoxication:		no			yes
Falling in love:	no			yes
Spreading the word: no			yes
God:				indifferent	is dead
Salvation here:		yes			no

\section{All about my mother}
So let's take Almodovar's movie All about my mother as an example of Juda advocate. In the movie we get one possible representation of a guy who reached nirvana from the Juda perspective, namely Rosa's father. He has an advanced Alzheimer's disease, he cant recognize Rosa anymore and his only question to a strangers are "What is your age" and "What is your height". So the scene when they meet, is when the two worlds meet. Rosa as true Juda, nun who is about to give birth (and also die at the same time) and her father[err ???]. But the magic in this scene is that the encounter works for both. He asks her about her age and height, she answers, and they are both satisfied. But there are two prerequisites for this scene to work. First is fathers dog, whom Rosa calls to the taxi and father then fallows and Manuela as a witness to this (big other?) Of course here is also a taxi driver, but he like all other true men in a movie are absolute non-agents/non-actors (don't have impact on the story) and thus can't count as an observer. So with the dogs ability to recognize and share his excitement with Rosa versus fathers inability we get the ultimate insult to the Buda. Her father stand here for the ultimate non-agent, less than unconscious animal. 
So some kind of conclusion here would be that if you are unable to fall, you are also unable to act. But there are two ways of falling. One manifest itself in the theater/spectacle (Barcelona) and other in stable (motherly) love (Madrid). And dividing her live between this two (passage between in movie is represented by a tunnel) is what at the end brings happiness to Manuella. So to be more precise, one way is falling into falling, attaching one self to attachment, kind of a spinning snow ball, embracing the world (here we are probably very close to Buda), and the other is an attachment to an object (son). What we don't get in Buda, is this back and forth movement. We only have before and after nirvana. But more importantly we have this ideal of a middle way.\footnote{What is also interesting is geographic position of Barcelona, as one of a few European cities facing east. There also one could mention Pattaya as a sort of eastern Asia's Barcelona; one of a few cities there facing west}
So it could be said that for Almodovar the main actors have obtained a kind of nirvana. Although they are still aching and loving, this does not stop them. They live in this samsara, where identities are fluid, nothing is certain and are nonetheless capable of keeping distance. (nice cul-de-sac scene) But what is really nice it's the effect that it has on a viewer, namely the simultaneous crying/laughter (tragicomedy par excellence). More precisely, we sympathize with actors (agents), but also acquire a distance from density of events. More precisely: The density disables us to emotionally process everything, and thus it pushes us to a far perspective, from where we are able to see a farce of a big picture.
[[(Almodovar: You are what you perform)]]
So one conclusion from film would be that Buda is about always taking the middle way, and Juda about oscillating between attachment and perpetual falling with a stoic/nirvana attitude.

So is Almodovar Buda after all? There is one aspect of a movie left, namely the moral of a story, who loses at the end. Loser is Nina. An only true addict (junkie) and in a way only "normal" woman in a film. (She gets married to a real man (non-actor), leaves the theater/Barcelona and gives birth to a ugly baby, thus she gets deprived of a stable/motherly love.) So her story stands as a reminder that there also exists the wrong kind of falling/attaching. Her attitude toward her own addiction is probably the strongest signifier: "I trade it for a bit of peace". In other words, I do it just to make everything a little easier (like a thing on a side) (Also a remark from Agrado: That fix didn't agree with you). So her problem is not embracing enough/trying to "dampen" a little everything. (Bad karma?)

So is Almodovar advocating Buda, or maybe some middle way between Buda (middle way) and Juda?  

(Half emptiness of a Manuella. If we want to be able to fall, we have to have a half empty; or another reading: This is the void left her son.
For Manuella, falling is bounded between two stable motherly loves.
Mother reproducing old artwork signifies stasis.)

(How to incorporate Buddhist notion of knowledge? Nina din't see her addiction for what it was: real addiction and a real threat to her relation with Huma. Here Agrado shows himself as a knowledgeable "guy", thus good.)

(Rosa; Even though she knew he could have aids, had sex with Esteban nonetheless. So its not that she was not knowledgeable, it was that she followed her hearth instead of mind. She uses mind afterwards, when its time to rationalize.. [Does Buda uses mind before? - but one should of course recall resistance story and how act is always to early/late...] )

%-----------------------
%=======================

-Of course we don't look at the process as having a temporal dimension. So as we speak of primordial singularity, the end (absolute knowledge) already happened, but still we must introduce the notion of a delay. (primordial?) 

\end{document}
